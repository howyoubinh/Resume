% Binh Nguyen Resume
% Compiled with LaTeX
% Updated 10/25/2020

% Git repo: https://github.com/howyoubinh/Resume

\documentclass{article}
    \usepackage{fullpage}
    \usepackage{amsmath}
    \usepackage{amssymb}
    \usepackage{textcomp}
    \usepackage[utf8]{inputenc}
    %\usepackage[rm,light]{roboto} % roboto light
    %\usepackage[sfdefault]{ClearSans} % Clear Sans
    %\usepackage[T1]{fontenc}
    \usepackage[margin=0.75in]{geometry}
    \usepackage{titlesec}
    \usepackage{titling}
    \usepackage{verbatim}
    \usepackage{hyperref}
    \usepackage{xcolor}
    \hypersetup{
    	     colorlinks,
    	     linkcolor={red!50!black},
    	     citecolor={blue!50!black},
    	     urlcolor={blue!80!black}
    	     }
    \textheight=10in
    \pagestyle{empty}
    \raggedright

\input{glyphtounicode}
\pdfgentounicode=1

\setlength{\parskip}{0pt}
\raggedbottom
\addtolength{\topmargin}{-4mm}

    %\renewcommand{\encodingdefault}{cg}
%\renewcommand{\rmdefault}{lgrcmr}

\def\bull{\vrule height 0.8ex width .7ex depth -.1ex }



% DEFINITIONS FOR RESUME %%%%%%%%%%%%%%%%%%%%%%%

\newcommand{\area} [2] {
    \vspace*{-9pt}
    \begin{verse}
        \textbf{#1}   #2
    \end{verse}
}

\newcommand{\lineunder} {
    \vspace*{-8pt} \\
    \hspace*{-18pt} \hrulefill \\
}

\newcommand{\header} [1] {
    {\hspace*{-18pt}\vspace*{6pt} \Large{#1} }
    \vspace*{-6pt} 
    \lineunder
}

\newcommand{\employer} [3] {
    { \textbf{#1} (#2)\\ \underline{\textbf{\emph{#3}}}\\  }
}

\newcommand{\contact} [3] {
    \vspace*{-10pt}
    \begin{center}
        {\Huge \scshape {#1}}\\
        #2 \\ #3
    \end{center}
    \vspace*{-8pt}
}

\newenvironment{achievements}{
    \begin{list}
        {$\bullet$}{\topsep 0pt \itemsep -2pt}}{\vspace*{4pt}
    \end{list}
}

\newcommand{\schoolwithcourses} [4] {
    \textbf{#1} #2 $\bullet$ #3\\
    #4 \\
    \vspace*{5pt}
}

\newcommand{\school} [4] {
    \textbf{#1} #2 $\bullet$ #3\\
    #4 \\
}

% !TeX spellcheck = en_US 

% END RESUME DEFINITIONS %%%%%%%%%%%%%%%%%%%%%%%

\begin{document}
\vspace*{-40pt}

  

%==== Profile ====%
%\vspace*{-10pt}
\begin{center}
	{\Huge \scshape {Binh Nguyen}}\\
	\vspace{2mm}
	\href{mailto:binhtnguyen95@gmail.com}{binhtnguyen95@gmail.com} $\cdot$ 
	209.406.6378 $\cdot$ 
	\href{https://www.linkedin.com/in/binh-t-nguyen}{linkedin.com/in/binh-t-nguyen} 
	%$\cdot$ 
	%\href{https://www.github.com/howyoubinh}{github.com/howyoubinh}\\
\end{center}

%==== Education ====%
\header{Education}
\textbf{University of California, San Diego}\hfill La Jolla, CA\\
B.S. Bioengineering: BioSystems, Minor: Cognitive Science \hfill 2018\\
\vspace{7mm}

%==== Skills ====%
\header{Skills}
\vspace{1mm}
\begin{tabular}{ l l }
%\hspace{-1mm}
	Programming: & Python, Matlab, C, JavaScript, SQL, UNIX, Shell, Git, Apache Subversion, \LaTeX \\
	Libraries:   & NumPy, Matplotlib, Pandas, Scikit-learn, Jupyter, TensorFlow, Keras, Django \\
	Systems/Hardware:    & LabVIEW, Simulink, Raspberry Pi, Arduino, PCB, Soldering, ECG \\
\end{tabular}
\vspace{5mm}

%==== Experience ====%
\header{Experience}
%\vspace{1mm}

\textbf{Acutus Medical, Inc.} \hfill Carlsbad, CA\\
\textit{Systems Engineer} \hfill August 2019 --- Present\\
\vspace{-2mm}
\begin{itemize} \itemsep 0.05pt
	\item Implemented pace blanking algorithm for correcting catheter positions in 3D localization engine
	\item Constructed and trained an RNN-LSTM model on clinical data to predict and detect disturbance
	\item Developed a real-time QRS detection method based on Pan Tompkins algorithm with over 95\% accuracy
	\item Generated respiration cycle templates for categorizing different types of cardiac arrhythmia
	\item Coordinated with cross-functional teams to execute V\&V testing for R\&D
\end{itemize}

\textit{Clinical Science Engineer Intern} \hfill July 2018 --- August 2019\\
\vspace{-2mm}
\begin{itemize} \itemsep 0.05pt
	\item Lead development of a segmentation algorithm for dividing left atrium into 8 distinct spatial regions for use as a clinical research tool
	\item Created visualization of propagation of conduction velocity vectors to identify localization of arrhythmic patterns
	\item Analyzed ECG signal using multi-modal methods to improve signal fidelity and localization accuracy
	\item Automated the retrieval, parsing, and organization of data from animal studies and clinical trials
\end{itemize}

%\vspace{-2mm}

\textbf{Cartilage Tissue Engineering Lab, UCSD} \hfill La Jolla, CA\\
\textit{Undergraduate Researcher} \hfill June 2017 --- August 2017\\
\vspace{-2mm}
\begin{itemize} \itemsep 0.05pt
	\item Validated the feasibility of automated cell counting against manual methods in human articular cartilage
	\item Reconstructed 3D tissue images from 2D cross-sectional images using Digital Volumetric Imaging in Matlab
	\item Collaborated with graduate students and lab faculty to implement 2D and 3D cell segmentation techniques involving adaptive thresholding and pixel intensities
	\item Presented research findings at the UCSD Summer Research Conference
\end{itemize}

%\vspace{-2mm}

\textit{Lab Assistant} \hfill August 2016 --- April 2018\\
\vspace{-2mm}
\begin{itemize} \itemsep 0.05pt
	\item Scanned, processed, and organized micro-CT images in local file system and server database
	\item Assisted with tissue culture generation, dissection, buffer-making, and staining for research experiments
	\item Checked, maintained, and serviced lab equipment to increase productivity and prevent downtime
	\item Wrote, conducted, and updated SOPs, ensuring up-to-date guidelines for lab-wide tasks
\end{itemize}
%\vspace{2mm}

%==== Projects ====%
%\iffalse
%\header{Projects}
%\textbf{Neural Network-controlled Drone via Voice Recognition} \hfill 2018
%\vspace{-2mm}
%\begin{itemize} \itemsep 0.05pt
%	\item Constructed DJI Flamewheel drone using DIY kit and interfaced with DroneKit Python API
%	\item Achieved 90\% accuracy in audio recognition using k-fold cross-validation methods via scikit-learn/TensorFlow
%\end{itemize}
%
%\textbf{Arduino-powered LED Pacemaker} \hfill 2017
%\vspace{-2mm}
%\begin{itemize} \itemsep 0.05pt
%	\item Designed ECG circuit with variable components to measure voltage and detect heart beats
%	\item Programmed an Arduino Mega to calculate heart rate and emit a different color LED based on the arrhythmia
%\end{itemize}
%\vspace{2mm}
%\fi
\vspace{5mm}
%==== Coursework ====%
\header{Coursework}
\vspace{1mm}
\begin{tabular}{ l l }
	Data Analysis, Machine Learning, Statistics and Probability, Data Structures, Algorithms \\
	Numerical Analysis, Analog Design, Circuits and Systems, Signal Processing, \\
	Bioinstrumentation, Biomedical Optics, Biomechanics, Human Physiology
\end{tabular}
\vspace{5mm}

%==== Interests ====%
\header{Interests}
Medical devices, HCI, Electrophysiology, VR, 3D Modeling, Music, Cooking, Fitness
\end{document}