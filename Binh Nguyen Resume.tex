% Binh Nguyen Resume
% Compiled with LaTeX
% Updated 10/25/2020

% Git repo: https://github.com/howyoubinh/Resume

\documentclass{article}
    \usepackage{fullpage}
    \usepackage{amsmath}
    \usepackage{amssymb}
    \usepackage{textcomp}
    \usepackage[utf8]{inputenc}
    %\usepackage[rm,light]{roboto} % roboto light
    %\usepackage[sfdefault]{ClearSans} % Clear Sans
    %\usepackage[T1]{fontenc}
%    \usepackage[margin=1in]{geometry}
    \usepackage[margin=0.75in]{geometry}
    \usepackage{titlesec}
    \usepackage{titling}
    \usepackage{verbatim}
    \usepackage{hyperref}
    \usepackage{xcolor}
% for UMichigan (start)
%    \usepackage{lipsum}
%    \usepackage{fancyhdr}
%    \usepackage[headheight=65pt,tmargin=65pt,headsep=20pt]{geometry}


%    \newcommand{\tmpx}{}
%    \newcommand\tmp[1]{\renewcommand{\tmpx}{#1}}
%    \fancypagestyle{sec}{\lhead{\tmpx}}
% for UMichigan (end)
    \hypersetup{
    	     colorlinks,
    	     linkcolor={red!50!black},
    	     citecolor={blue!50!black},
    	     urlcolor={blue!80!black}
    	     }
    \textheight=10in
    \pagestyle{empty}
    \raggedright

\input{glyphtounicode}
\pdfgentounicode=1

\setlength{\parskip}{0pt}
\raggedbottom
%\addtolength{\topmargin}{-4mm}

    %\renewcommand{\encodingdefault}{cg}
%\renewcommand{\rmdefault}{lgrcmr}

%\def\bull{\vrule height 0.8ex width .7ex depth -.1ex }


%\documentclass{article} 
%\usepackage[margin=1in]{geometry}

% set margins to 1 inch
%\setlength\topmargin{0pt}
%\addtolength\topmargin{-\headheight}
%\addtolength\topmargin{-\headsep}
%\setlength\oddsidemargin{0pt}
%\setlength\textwidth{\paperwidth}
%\addtolength\textwidth{-2in}
%\setlength\textheight{\paperheight}
%\addtolength\textheight{-2in}
%\usepackage{layout}


% DEFINITIONS FOR RESUME %%%%%%%%%%%%%%%%%%%%%%%

\newcommand{\area} [2] {
    \vspace*{-9pt}
    \begin{verse}
        \textbf{#1}   #2
    \end{verse}
}

\newcommand{\lineunder} {
    \vspace*{-8pt} \\
    \hspace*{-18pt} \hrulefill \\
}

\newcommand{\header} [1] {
    {\hspace*{-18pt}\vspace*{6pt} \Large{#1} }
    \vspace*{-6pt} 
    \lineunder
}

\newcommand{\employer} [3] {
    { \textbf{#1} (#2)\\ \underline{\textbf{\emph{#3}}}\\  }
}

\newcommand{\contact} [3] {
    \vspace*{-10pt}
    \begin{center}
        {\Huge \scshape {#1}}\\
        #2 \\ #3
    \end{center}
    \vspace*{-8pt}
}

\newenvironment{achievements}{
    \begin{list}
        {$\bullet$}{\topsep 0pt \itemsep -2pt}}{\vspace*{4pt}
    \end{list}
}

\newcommand{\schoolwithcourses} [4] {
    \textbf{#1} #2 $\bullet$ #3\\
    #4 \\
    \vspace*{5pt}
}

\newcommand{\school} [4] {
    \textbf{#1} #2 $\bullet$ #3\\
    #4 \\
}


% !TeX spellcheck = en_US 

% END RESUME DEFINITIONS %%%%%%%%%%%%%%%%%%%%%%%

    

\begin{document}
%\pagestyle{fancy}
%\fancyhf{}
%\renewcommand{\headrulewidth}{0pt}
%\lhead{Curriculum Vitae/Resume \\ Binh Nguyen \\ Biomedical Engineering M.S. \\ 65375374}

% this controls the whitespace before name
\vspace*{-50pt}


%==== Profile ====%
%\vspace*{-10pt}
\begin{center}
	{\Huge \scshape {Binh Nguyen}}\\
	\vspace{2mm}
	\href{mailto:binhtnguyen95@gmail.com}{binhtnguyen95@gmail.com} $\cdot$ 
	209.406.6378 $\cdot$ 
	\href{https://www.linkedin.com/in/binh-t-nguyen}{linkedin.com/in/binh-t-nguyen} 
	%$\cdot$ 
	%\href{https://www.github.com/howyoubinh}{github.com/howyoubinh}\\
\end{center}

%==== Education ====%
\header{Education}
\textbf{University of California, San Diego}\hfill La Jolla, CA\\
B.S. Bioengineering: BioSystems, Minor: Cognitive Science \hfill 2018\\
\vspace{7mm}

%==== Professional and Research Experience ====%
\header{Professional and Research Experience}
%\vspace{1mm}

\textbf{Acutus Medical, Inc.} \hfill Carlsbad, CA\\
\textit{Software \& Systems Quality Engineer II} \hfill March 2022 --- Present\\
\vspace{-2mm}
\begin{itemize} \itemsep 0.05pt
	\item Implemented automated regression testing for WPF applications using a modified XPath language
	\item Improved efficiency of quality compliance for thousands of BOM items using Selenium web automation
	\item Drafted and executed software test cases for multiple projects and produced detailed bug reports in Jira
	\item Supported cross-functional teams by developing software quality test plans using Scrum methodologies
\end{itemize}

\textit{R\&D Systems Engineer} \hfill August 2019 --- March 2022\\
\vspace{-2mm}
\begin{itemize} \itemsep 0.05pt
	\item Built prototyping tools to integrate 3D magnetic tracking capability for next-generation systems
	\item Developed MATLAB application for reading, monitoring, and displaying real-time multi-modal UDP data
	\item Trained LSTM models in TensorFlow to detect and predict disturbances in bioimpedance signals
%	\item Established template-matching methodology to classify and compensate for anomalous respiration cycles
	\item Performed root-cause analysis on electro-mechanical systems using digital signal processing techniques
	\item Standardized production systems using clinical site data and reduced number of complaint reports
%	\item Coordinated with hardware, firmware, software, and quality teams to execute V\&V testing
\end{itemize}

\textit{Clinical Science Engineer Intern} \hfill July 2018 --- August 2019\\
\vspace{-2mm}
\begin{itemize} \itemsep 0.05pt
	\item Developed semi-automatic 3D heart segmentation algorithm for statistical and clinical analysis
	\item Created MATLAB visualization of conduction velocity vectors to identify repetitive arrhythmic patterns
	\item Automated the retrieval, parsing, and organization of data from animal studies and clinical trials
	\item Gained unique perspective on how to turn theoretical algorithms into useful clinical applications
%	\item Optimized ECG signal using multi-modal methods to improve signal-to-noise ratio
\end{itemize}

%\vspace{-2mm}
%\vspace{5mm}

%==== Research Experience ====%
%\header{Research Experience}
\textbf{Mathematical Neuroscience Lab, UCSD} \hfill La Jolla, CA\\
\textit{Senior Design Project Team Member} \hfill September 2017 --- June 2018\\
\vspace{-2mm}
\begin{itemize} \itemsep 0.05pt
	\item Worked with team members to plan, design, implement, and analyze a new class of dynamic ANNs
	\item Constructed DJI Flamewheel drone using DIY kit and leveraged DroneKit Python API and Raspberry Pi
	\item Achieved 90\% accuracy in audio recognition task using k-fold cross-validation methods via TensorFlow
	
\end{itemize}

\textbf{Cartilage Tissue Engineering Lab, UCSD} \hfill La Jolla, CA\\
\textit{Undergraduate Researcher} \hfill June 2017 --- August 2017\\
\vspace{-2mm}
\begin{itemize} \itemsep 0.05pt
	\item Reconstructed 3D tissue images from 2D cross-sectional images using Digital Volumetric Imaging in MATLAB
	\item Collaborated with graduate students and lab faculty to implement 2D and 3D cell segmentation techniques
	\item Elucidated cell variability in superficial and deep zones and found disparities in manual cell counting
	\item Validated the feasibility of automated cell counting against manual methods in human articular cartilage
	\item Presented research findings at the 2017 UCSD Summer Research Conference to diverse audience members
\end{itemize}

%\vspace{-2mm}

\textit{Lab Assistant} \hfill August 2016 --- April 2018\\
\vspace{-2mm}
\begin{itemize} \itemsep 0.05pt
	\item Conducted tissue culture generation, dissection, staining, and imaging
	\item Processed and analyzed micro-CT images of bovine cartilage samples
	\item Followed and revised Standard Operating Procedures to ensure safety and quality standards
	\item Maintained lab equipment and freezers through daily and weekly inspections and measurements
\end{itemize}

%\vspace{2mm}

%==== Projects ====%
%\iffalse
%\header{Projects}
%\textbf{Neural Network-controlled Drone via Voice Recognition} \hfill 2018
%\vspace{-2mm}
%\begin{itemize} \itemsep 0.05pt
%	\item Constructed DJI Flamewheel drone using DIY kit and interfaced with DroneKit Python API
%	\item Achieved 90\% accuracy in audio recognition using k-fold cross-validation methods via scikit-learn/TensorFlow
%\end{itemize}
%
%\textbf{Arduino-powered LED Pacemaker} \hfill 2017
%\vspace{-2mm}
%\begin{itemize} \itemsep 0.05pt
%	\item Designed ECG circuit with variable components to measure voltage and detect heart beats
%	\item Programmed an Arduino Mega to calculate heart rate and emit a different color LED based on the arrhythmia
%\end{itemize}
%\vspace{2mm}
%\fi
\vspace{5mm}

%==== Technical Skills ====%
\header{Technical Skills}
\vspace{1mm}
\begin{tabular}{ l l }
%\hspace{-1mm}
	Programming Languages: & Python, MATLAB, C/C++, SQL, UNIX, Git, \LaTeX \\
	Software Libraries:   & Scikit-learn, TensorFlow, Keras, Jupyter, Selenium\\
	Prototyping Tools:    & LabVIEW, Simulink, Raspberry Pi, Arduino, PCB, Soldering  \\
\end{tabular}
\vspace{5mm}

%==== Coursework ====%
%\header{Coursework}
%\vspace{1mm}
%\begin{tabular}{ l l }
%	Python for Data Analysis, Machine Learning, Data Structures, Algorithms, Statistics and Probability\\
%	Numerical Analysis, Analog Design, Circuits and Systems, Signal Processing \\
%	Bioinstrumentation, Biomedical Optics, Biomechanics, Human Physiology
%\end{tabular}
%\vspace{5mm}

%==== Interests ====%
%\header{Interests}
%Data science, Human-computer interaction, Electrophysiology, Neuroscience, Automation, Sequencing, DSP
\end{document}